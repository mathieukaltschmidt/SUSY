% !TEX root = ../summary_jonah.tex
% !TEX program = xelatex
% !TEX encoding = utf8
% !TEX spellcheck = en_UK

\section{Introduction}
Since its proposal, Supersymmetry (SUSY) has enabled generations of physicists to study the ramifications of field theory on fundamental scales and gain insight in and a better understanding of the intrinsic framework of nature herself. Naturally, it would be appealing to find some extension of the ordinary Standard Model – it being one of the most successful models of modern day particle physics – in the sumptuous pool of new models found in supersymmetry. The most simple model, constructible by extending the Standard Model supersymmetrically, is the Minimal Supersymmetric Standard Model in which all the previous fields are promoted to superfields. Before a detailed discussion of the mathematical structure of the MSSM in section~\ref{mssm-basics} can be pursued, the reader will be made acquainted with the basic construction of the Standard Model and its parameters in section~\ref{sm-basics}. From there on, the MSSM description follows quite straight-forwardly. More delicate details like the mixing in the particle spectrum will be discussed more thoroughly in section~\ref{mssm-details}. Some outlook on how to implement beyond the Standard Model physics will be given in section~\ref{disc}.