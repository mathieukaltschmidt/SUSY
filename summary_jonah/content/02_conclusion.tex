% !TEX root = ../summary_jonah.tex
% !TEX program = xelatex
% !TEX encoding = utf8
% !TEX spellcheck = en_UK

\section{Discussion and outlook}\label{disc}
Following the analysis given here, the free parameter count of the Standard Model is richly enhanced by applying SUSY and enforcing SUSY breaking explicitly. Unfortunately, such a large parameter space is just too complex to be tractable and impossible to navigate in the scope of making predictions that can be falsified by experiment. Furthermore, many of the new parameters introduce effects like unsuppressed Flavour Changing Neutral Currents (FCNC), violation of the generational lepton number conservation, and additional CP violation. All these effects are heavily constrained by experimental bounds. In most applications of the MSSM this is the key reason why the parameters responsible for these processes are excluded. Additionaly, mixing in the sfermion sector often is restricted to the third generation only, to further reduce the parameter count. One example for such a simplification would be the phenomenological MSSM (pMSSM), where only 19 additional parameters remain and it becomes significantly easier to make testable predictions. Other models assume gauge coupling unification at higher scales, like the minimal supergravity model (mSUGRA). In these cases the description with fewer parameter holds at the unification scale and by using the renormalisation group flow the rich phenomenology of the MSSM can be recovered at lower scales. Another interesting thing about supergravity would be that the naturally emerging superpartner of the graviton $G$, the gravitino $\tilde{G}$, could be a candidate for the LSP. If the supersymmetry in consideration acts locally, the SUSY breaking would lead to the emergence of a goldstino $\tilde{G}_{\frac12}$, which often becomes the LSP. In SUGRA the Goldstino could be eaten by the Gravitino and would contribute to its mass $m_{\frac32}$~\cite{pdg, primer}.\\
Clearly, the story must not end here, since many of the still unexplained aspects of the SM more or less directly translate into new problems in the MSSM. To name one example, the mystery of the neutrino mass still remains unsolved in the MSSM, albeit proposals like the seesaw mechanism can be implemented in an MSSM-consistent fashion~\cite{pdg}. As another issue still present in the MSSM, one could name the strong CP-problem, whose axionic solution can also be incorporated supersymmetrically, leading to other candidates for LSPs~\cite{pq, primer, chung}.\\
And of course, just by adding additional fields or relaxing symmetry requirements, there are no limitations on the lengths to which certain theoretical endeavors can go. In the end, the MSSM is an importand stepping stone in paving the way towards a (more) complete understanding of nature at fundamental scales.