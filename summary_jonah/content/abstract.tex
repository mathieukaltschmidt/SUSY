% !TEX root = ../summary_jonah.tex
% !TEX program = xelatex
% !TEX encoding = utf8
% !TEX spellcheck = en_UK

\begin{center}
	\makeatletter
	\thispagestyle{plain}
	\LARGE\textbf{\@title} \\
	\vspace{2mm}
	\Large\textbf{\@subtitle} \\
	\vspace{2mm}
	\large\bfseries{\@author} \\
	\normalfont
	\vspace{2mm}
	\large{\@date} \\
	\vspace{2mm}
	\large{Institute for Theoretical Physics \\
		Heidelberg University} \\
	\makeatother
\end{center}

\normalsize
\noindent This report summarizes the contribution to a talk presented at the SUSY Seminar organized by Prof. J\"org J\"ackel at the Institute for Theoretical Physics in Heidelberg during the winter term 2020/2021 together with Mathieu Kaltschmidt.\\
First, the framework of the Standard Model of particle physics is reviewed. Afterwards, the construction of the minimal supersymmetric extension, the Minimal Supersymmetric Standard Model, is presented.
The field content and its embedding in a Lagrangian description is given. Special emphasis is laid on certain phenomenological aspects like the tree-level Higgs masses and the mixing of the Standard Model superpartners in the richly enhanced particle spectrum of the MSSM. For both models a detailed parameter count is performed leading to the well-known 19 parameters for the Standard Model and the MSSM-124.
To close the discussion, some examples of models with greatly reduced parameter space, as well as starting points for embedding the MSSM in more general theories like e.\ g.\ Super Gravity, are given.\\
Familarity with the basic concepts and structure of SUSY will be assumed throughout the discussion.