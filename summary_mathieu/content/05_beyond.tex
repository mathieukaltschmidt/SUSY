\section{Going Beyond the MSSM}
As the name already suggests, the \enquote{minimal} supersymmetric extension of the Standard Model may not be enough to solve all our remaining problems. Many extensions of the MSSM have been proposed to deal with various open questions such as the missing explanation of the fundamental origin of SUSY breaking, the \enquote{total} particle content, the explicit structure of $\mathcal{G}_{\mathrm{MSSM}}$\footnote{In the context of GUTs there has been a lot of research on the embedding of the Standard Model gauge group into larger groups such as $\operatorname{SO}(10)$ which facilitates the inclusion of neutrinos or even more complicated groups such as $E_6$ or $E_8\times E_8$ motivated by the study of String Theory \cite{Langacker2012}.} and many more. \\ To conclude this discussion we want to present some conceptual ideas on the explanation of the value of the Higgs-higgsino mass parameter $\mu$ to get a feeling on how model building in the context of the MSSM works and how those different ideas lead to further complexity in the resulting parameter spaces.\\
The general problem with $\mu$ is that it is actually a SUSY preserving parameter but from phenomenological constraints we know that it must be of the order of the SUSY breaking scale.  A natural way of dealing with this discrepancy would be a symmetry enforcing $\mu=0$, and a small SUSY breaking parameter generating a value of $\mu$ that is not parametrically larger than the breaking scale.\\
In proposed extensions of the MSSM a first idea would be to replace $\mu$ by the vacuum expectation value of a new $\operatorname{SU}(3)\times\operatorname{SU}(2)\times\operatorname{U}(1)$ scalar singlet. This is often referred to as the next-to-minimal SSM (NMSSM). Another possibility would be the addition of a new broken $\operatorname{U}(1)$ gauge symmetry to  the NMSSM, which is then called the $\operatorname{U}(1)$-extended SSM or USSM for short \cite{Cvetic1997}. New mass terms analogous to the $\mu$ term in the MSSM could be introduced by allowing all renormalizable terms in the superpotential (\enquote{generalized} NMSSM) \cite{Ross2012}. \\
Interesting connections to modern research on Axion physics may provide a solution to the strong-CP problem and solve the $\mu$ problem at the same time. Such models include additional gauge singlets which are charged under the so called Peccei-Quinn symmetry. The breaking of the PQ symmetry may then be connected to SUSY breaking and would yield values of $\mu$ of the order of the electroweak scale \cite{Peccei2006}. \\
For work on higher dimensional Higgs multiplets we refer for example to \cite{Delgado2012}. Here, the concept of custodial symmetry which is used to test the electroweak sector of the SM with high precision plays a central role. Such models may provide interesting phenomenological implications for future experimental studies. \\
Even if most of these publications we referred to have already been published a few years ago, these (short) considerations already show that beyond the MSSM studies go into various interesting directions and even if large areas of the possible MSSM parameter space are already ruled out by experiment, there might be a chance that SUSY is realized in Nature and provides a beautiful explanation of many fascinating phenomena.
\newpage