\section{Conclusion and Outlook}
Getting access to the rich phenomenology of the MSSM has turned out to be a very challenging task due to the complexity of the underlying parameter space with its 124 relevant parameters in the simplest version of the MSSM. \\
Using very general phenomenological constraints on flavor-changing neutral currents and CP-violation which are tested with high precision by various experiments, we were able to reduce the number of relevant parameters in the MSSM already by a large amount. \\ 
The fact that the MSSM does not provide an explanation for the origin of SUSY breaking can be seen as a huge problem at first but it also allowed us to develop the mSUGRA framework in the second part of our talk assuming gravity-mediated SUSY breaking at some high energy scale and a minimal structure of the kinetic terms of the SUGRA Lagrangian. From this we were able to construct a model for low-energy MSSM phenomenology resulting from a renormalization group approach starting with a relatively simple structure of the theory at high energies. \\
We then discussed the concept of gauge coupling unification in more detail in the context of the Standard Model. We presented the explicit formulas for the running of the (inverse) couplings and how the superpartners in the MSSM change this result. \\
To conclude our talk we presented some ideas on beyond the MSSM models that try to explain some of the remaining open questions such as the value of the $\mu$ parameter. \\
Of course, there have been extensive studies and experimental tests of SUSY phenomenology in the last couple of  years. Unfortunately it has turned out, that up to now no promising signatures of SUSY have been detected by experiments and we therefore have to put even more severe constraints on new models. Maybe these new constraints will help us in the end to finally detect SUSY signatures in future experiments or to ultimately rule out the possibility of a realization of SUSY in Nature. 