% !TEX root = ../main.tex
% !TEX program = xelatex
% !TEX encoding = utf8
% !TEX spellcheck = en_UK

\section{The Standard Model}
\subsection{Basics}
\begin{frame}{Standard Model — The Basics\only<5>{ (Well, more or less...)}}
\only<1, 2, 4>{The gauge group of the Standard Model (SM) is
\begin{align*}
  \mathcal{G}_\mathrm{SM}&=\su(3)_C\times\su(2)_T\times\un(1)_Y.
\end{align*}}
\only<3>{
\alert{Beware!} We only use lefthanded fields $\psi_L=\vect{\psi_\alpha\\0}$, righthanded fields are included via charge conjugation $\left(\psi_R\right)^c=\vect{0\\\bar{\psi}^{\dot{\alpha}}}^c=\vect{\psi_\alpha\\0}.$
\vfill
The $\su(2)$-doublets are written as:
\begin{align*}
\phi=\vect{\phi_+\\\phi_0}\quad\quad q_L=\vect{u_L\\d_L}\quad\quad\ell_L=\vect{\nu_L\\e_L}.
\end{align*}
}
\only<2, 4>{\vfill
The SM fields transform in representations of this group:
\begin{align*}
s&=0:\quad\underset{\ni\phi}{\rep{1}{2}{\frac12}}\\
s&=\frac12:\quad\underset{\ni q_L}{\rep{3}{2}{\frac16}}\oplus\underset{\ni(d_R)^c}{\crep{3}{1}{\frac13}}\oplus\underset{\ni(u_R)^c}{\crep{3}{1}{-\frac23}}\oplus\underset{\ni\ell_L}{\rep{1}{2}{-\frac12}}\oplus\underset{\ni(e_R)^c}{\rep{1}{1}{1}}\\
s&=1:\quad\underset{\ni{A}^a_\mu}{\rep{8}{1}{0}}\oplus\underset{\ni W^k_\mu}{\rep{1}{3}{0}}\oplus\underset{\ni B_\mu}{\rep{1}{1}{0}}.
\end{align*}
}
\only<5>{\alert{\textsc{Georgi-Glashow} model:} Using the breaking pattern \mbox{$\su(5)\to\mathcal{G}_\mathrm{SM}$} the previous fields fit nicely into representations of \mbox{$\su(5)$}:
\begin{align*}
  \bar{\mathbf{5}}&=\crep{3}{1}{\frac13}\oplus\rep{1}{2}{-\frac12}\\
  \mathbf{10}&=\rep{3}{2}{\frac16}\oplus\crep{3}{1}{-\frac23}\oplus\rep{1}{1}{1}\\\\
  \mathbf{24}&=\rep{8}{1}{0}\oplus\rep{1}{3}{0}\oplus\rep{1}{1}{0}\oplus\rep{3}{2}{-\frac53}\oplus\crep{3}{2}{\frac53}.
\end{align*}
}
\end{frame}
\subsection{The Lagrangian}
\begin{frame}{Standard Model — The Lagrangian}
The SM Lagrangian contains a variety of terms which roughly fall into three categories:\\
\vfill
\alert{gauge terms\hfill kinetic terms\hfill Higgs sector\hfill\hfill}
\vfill
The gauge part is straightforward albeit there being additional gauge configurations:
\begin{align*}
  \mathcal{L}_\text{gauge}&\supset\frac1{2g_i^2}\tr\left[F^{(i)}_{\mu\nu}F^{(i)\mu\nu}\right]\\
  &\supset\frac{\theta_\mathrm{QCD}}{16\pi^2g_\mathrm{s}^2}\epsilon^{\mu\nu\rho\sigma}\tr\left[F^{(3)}_{\mu\nu}F^{(3)}_{\rho\sigma}\right].
\end{align*}
Kinetic terms for the fermions are constructed with the covariant derivative
\begin{align*}
  \mathcal{L}_\mathrm{kin}&\supset\bar{\psi}_i\imag\slashed{D}\psi_i\\
  D_\mu&=\partial_\mu-\imag qA^k_\mu\mathcal{R}(T_k).
\end{align*}
\end{frame}

\begin{frame}{Standard Model — Higgs sector}
The Higgs part is made of a scalar kinetic term, the quartic potential and the Yukawa couplings:
\begin{align*}
  \mathcal{L}_\mathrm{Higgs}\supset-\left(D^\mu\phi\right)^\dagger\left(D_\mu\phi\right)+\mu\phi^\dagger\phi-\lambda\left(\phi^\dagger\phi\right)^2-\left(\lambda^\psi\left[\bar{\psi}\phi\psi\right]_\numb{1}+\hc\right).
\end{align*}
\only<2, 3>{
The fermions gain masses \mbox{$m_\psi=v\lambda^\psi$} when the Higgs field acquires its vacuum expectation value (VEV):
\begin{align*}
  \left\langle\phi\right\rangle&=\left(\begin{array}{c}0\\v\end{array}\right)\kern4em\color{black}{\left(\text{shorthand: }\color{DarkBlue}{\left\langle\phi_0\right\rangle=v}\color{black}{}\right)},
\end{align*}
while the Higgs mass becomes \mbox{$m_h=2\sqrt{\lambda}v$}.\\
}
\only<3, 4>{
For our three types of massive fermions (electron, up-quark, down-quark) the corresponding singlets look like:\footnote{Careful, since $\tilde{\phi}=\epsilon\phi^*$ to account for the u-type quarks.}
\begin{align*}
\lambda^u\left[\bar{q}_L\tilde{\phi}u_R\right]_\numb{1},\quad\quad\lambda^d\left[\bar{q}_L\phi d_R\right]_\numb{1},\quad\quad\lambda^e\left[\bar{\ell}_L\phi e_R\right]_\numb{1}.
\end{align*}
\only<4>{Furthermore, the SM fermion fields consists of three generations, thus promoting the $\lambda^\psi$ to complex 3x3 matrices $\lambda^e_{mn}$, $\lambda^d_{mn}$ and $\lambda^u_{mn}$. These $\lambda^f_{mn}$ can be diagonalised via bi-unitary transformations:
\begin{align*}
  V_f^\dagger\lambda^fU_f&\propto\operatorname{diag}\left(m^{(1)}_f, m^{(2)}_f, m^{(3)}_f\right)/v.
\end{align*}
}
}
\end{frame}
\subsection{Parameter Count}
\begin{frame}{Standard Model — Parameter count}
Count parameters and gauge redundancies:
\vfill
\only<1, 2, 3>{$+$
\begin{itemize}
\item 3 couplings $g$, $g'$ and $g_\mathrm{s}$, one vacuum angle $\theta_\mathrm{QCD}$ (4 parameters)
\item Higgs parameters $v$, $\lambda$ (2 parameters)
\item 3 (complex) mass matrices $\lambda^f$ (3x18 parameters)
\end{itemize}
}\only<2, 3>{$-$
\begin{itemize}
\item Quark flavour symmetry $\un(3)_{q_L}\times\un(3)_{u_R}\times\un(3)_{d_R}/\un(1)_B$ (3x9-1 parameters)
\item Lepton flavour symmetry $\un(3)_{\ell_L}\times\un(3)_{e_R}/\un(1)_{L_e}\times\un(1)_{L_\mu}\times\un(1)_{L_\tau}$ (2x9-3 parameters)
\end{itemize}
}\vfill
\only<3>{The Standard Model of Particle Physics has \alert{19 free parameters} with $v$ being the only one carrying a physical dimension.\footnote{14 real parameters, 3 mixing angles, 2 CP-violating phase.}}
\end{frame}
\section{The Minimal Supersymmetric Standard Model}
\subsection{Basics}
\begin{frame}{MSSM — The fields}
First of all we promote all our previous fields to real (chiral) superfields resulting in our renewed table:\vfill
\begin{table}
\bgroup
\def\arraystretch{1.4}
\begin{tabular}{cccl}
  super field& bosonic field& fermionic field& representation\\\hline
  $\hat{V}_8$& $g$& $\tilde{g}$& $\quad\quad\rep{8}{1}{0}$\\
  $\hat{V}$& $W^0$, $W^\pm$& $\tilde{W}^0$, $\tilde{W}^\pm$& $\quad\quad\rep{1}{3}{0}$\\
  $\hat{V}'$& $B$& $\tilde{B}$& $\quad\quad\rep{1}{1}{0}$\\
  $\hat{L}$& $(\tilde{\nu}_L, \tilde{e}_L)$& $(\nu_L, e_L)$& $\quad\quad\rep{1}{2}{-\frac12}$\\
  $\hat{E}^c$& $\tilde{e}_R^c$& $e_R^c$& $\quad\quad\rep{1}{1}{1}$\\
  $\hat{Q}$& $(\tilde{u}_L, \tilde{d}_L)$& $(u_L, d_L)$& $\quad\quad\rep{3}{1}{\frac16}$\\
  $\hat{U}^c$& $\tilde{u}_R^c$& $u_R^c$& $\quad\quad\rep{3}{1}{-\frac23}$\\
  $\hat{D}^c$& $\tilde{d}_R^c$& $d_R^c$& $\quad\quad\rep{3}{1}{\frac13}$\\
  $\hat{H}_u$& $\left(H_u^+, H_u^0\right)$& $\left(\tilde{H}_u^+, \tilde{H}_u^0\right)$& $\quad\quad\rep{1}{2}{\frac12}$\\
  $\hat{H}_d$& $(H_d^0, H_d^-)$& $(\tilde{H}^0_d, \tilde{H}_d^-)$& $\quad\quad\rep{1}{2}{-\frac12}$
\end{tabular}
\egroup
\end{table}
\pause
\alert{Beware!} We need a second Higgs doublet to cancel the gauge anomaly introduced by the Higgsinos!
\end{frame}

\subsection{The Lagrangian — SUSY conserving}
\begin{frame}{MSSM — SUSY terms}
For the gauge part the ususal field strength super fields
\begin{align*}
\mathcal{W}_{i, \alpha}=-\frac14\bar{D}^2e^{-\hat{V}}D_\alpha e^{\hat{V}},
\end{align*}
are constructed and included in the Lagrangian:
\begin{align*}
  \mathcal{L}^\mathrm{MSSM}_\mathrm{gauge}\supset&\frac1{2g_i^2}\tr\left[\int\dd^2\theta\left(\mathcal{W}_i\right)^\alpha\left(\mathcal{W}_i\right)_\alpha+\hc\right].
\end{align*}
The kinetic terms for the fields read:
\begin{align*}
  \mathcal{L}^\mathrm{MSSM}_K\supset&\int\dd^2\theta\dd^2\bar{\theta}\left[\hat{\Phi}^\dagger_ie^{2V_i}\hat{\Phi}_i\right]_\numb{1}\\
  V_i=&\hat{V}_8^a\mathcal{R}_i(T_a)+\hat{V}^k\mathcal{R}_i(T_k)+Y_i\hat{V}'.
\end{align*}
\end{frame}

\begin{frame}{MSSM — Superpotential}
The superpotential term is simply:
\begin{align*}
  \mathcal{L}^\mathrm{MSSM}_W=&\int\dd^2\theta W+\hc
\end{align*}
\only<1>{What terms are contained in $W$?}\only<2, 3, 4>{In general, the superpotential contains a great variety of different terms, under them the Yukawa couplings:
\begin{align*}\nonumber
  W=&\lambda_d\left[\hat{H}_d\hat{Q}\hat{D}\right]_\numb{1}+\lambda_e\left[\hat{H}_d\hat{L}\hat{E}\right]_\numb{1}-\lambda_u\left[\hat{H}_u\hat{Q}\hat{U}\right]_\numb{1}+\mu \left[\hat{H}_u\hat{H}_d\right]_\numb{1}\\
  &+a\left[\hat{L}\hat{H}_u\right]_\numb{1}+b\left[\hat{Q}\hat{L}\hat{D}\right]_\numb{1}+c\left[\hat{U}\hat{U}\hat{D}\right]_\numb{1}+d\left[\hat{L}\hat{L}\hat{E}\right]_\numb{1}.
\end{align*}
}\only<3>{The terms in the last line introduce $B$-number violation via proton decay as well as lepton number violation, but by imposing \alert{$\color{HDRed}{R}$-parity}
\begin{align*}
  R=(-1)^{3(B-L)+2s},
\end{align*}
we can get rid of them. \alert{Beware!} This is not obligatory!

$R$-conservation implies the existance of a lightest supersymmetric particle (LSP) thus providing us with a dark matter candidate.
}\only<4>{We could use \alert{matter parity}
\begin{align*}
P_\mathrm{m}&=(-1)^{3(B-L)},
\end{align*}
instead and see directly how the lower line gets thrown out.
}
\end{frame}
\subsection{The Lagrangian — SUSY breaking}
\begin{frame}{MSSM — Soft SUSY breaking terms}
Introduce explicitly SUSY breaking terms to generate masses and additional interactions
\begin{align*}\nonumber
  -\mathcal{L}^\mathrm{MSSM}_\mathrm{soft}\supset&\frac12M_i\tilde{\bar{\lambda}}_i\tilde{\lambda}_i+M^2_{\tilde{F}}\tilde{f}^\dagger\tilde{f}\\
  &+m_1^2H_d^\dagger H_d+m_2^2H_u^\dagger H_u+m_{12}^2\left(H_u\cdot H_d+\hc\right)\\\nonumber
  &+T_UH_u\tilde{Q}\tilde{U}+T_DH_d\tilde{Q}\tilde{D}+T_EH_d\tilde{L}\tilde{E}+\hc
\end{align*}
Often, a parametrisation $m_{12}^2=\mu B$ (and $T_F=\lambda_f A_F$) is chosen. Therefore, the corresponding terms are called \alert{A} and \alert{B-terms}.
\end{frame}
\subsection{(Effective) Higgs potential}
\begin{frame}{MSSM — Note on the Higgs sector}
Repeat the SM steps:\\
In the MSSM the quartic coupling is generated by the \alert{D-terms} of the Kähler potential, and the SUSY breaking terms, leading to an effective Higgs potential.
\begin{align*}\nonumber
  V_\mathrm{Higgs}=&\left(m_1^2+|\mu|^2\right)H_d^\dagger H_d+\left(m_2^2+|\mu|^2\right)H_u^\dagger H_u+m_{12}^2\left(H_u\cdot H_d+\hc\right)\\
  &+\frac{g^2+{g'}^2}8\left(H_d^\dagger H_d-H_u^\dagger H_u\right)+\frac12g^2\left|H_d^\dagger H_u\right|^2,
\end{align*}
\only<2>{The two doublets acquire separate VEVs
\begin{align*}
  \left\langle H_f^0\right\rangle=v_f,
\end{align*}
related to the previous $v$ via
\begin{align*}
  \sqrt{v_u^2+v_d^2}&=v,
\end{align*}
by convention, the angle $\beta$ is defined as
\begin{align*}
  \tan\beta&=\frac{v_u}{v_d}.
\end{align*}
}
\only<3, 4>{At tree level this implies an \alert{upper bound} on the mass of the lightest Higgs:
\begin{align*}
  m^2_h&\le m^2_Z\cos^22\beta\only<4>{+\cdots}.
\end{align*}
}
\end{frame}
\subsection{Particle spectrum}
\begin{frame}{MSSM — Mixing caveats}
\only<1-5>{The SM particle spectrum looks like:
\vspace{-1em}}\bgroup\large
\only<1>{
\begin{table}
{\setlength{\extrarowheight}{5pt}
\begin{tabular}{cccccc}
$u$& $c$& $t$&& $\color{red}{B}$& $\color{red}{W^0}$\\
$d$& $s$& $b$&& $g$& $W^\pm$\vspace{.75em}\\
$e$& $\mu$& $\tau$&& $h^0$&\\
$\nu_e$& $\nu_\mu$& $\nu_\tau$&&&
\end{tabular}
}\end{table}
}
\only<2>{
\begin{table}
{\setlength{\extrarowheight}{5pt}
\begin{tabular}{cccccc}
$u$& $c$& $t$&& $\color{red}{B}$& $\color{red}{W^0}$\\
$d$& $s$& $b$&& $g$& $W^\pm$\vspace{.75em}\\
$e$& $\mu$& $\tau$&& $\color{purple}{H_d^-}$&$\color{orange}{H_d^0}$\\
$\nu_e$& $\nu_\mu$& $\nu_\tau$&&$\color{orange}{H_u^0}$&$\color{purple}{H_u^+}$
\end{tabular}
}\end{table}
}
\only<3, 4, 5>{
\begin{table}
{\setlength{\extrarowheight}{5pt}
\begin{tabular}{cccccc}
$u$& $c$& $t$&& $\color{red}{\gamma}$& $\color{red}{Z^0}$\\
$d$& $s$& $b$&& $g$& $W^\pm$\vspace{.75em}\\
$e$& $\mu$& $\tau$&& $\color{orange}{h^0}$&$\color{orange}{H^0}$\\
$\nu_e$& $\nu_\mu$& $\nu_\tau$&&$A^0$&$\color{purple}{H^\pm}$
\end{tabular}
}\end{table}
}
\egroup
\only<4, 5>{The MSSM particle spectrum\only<5>{\footnote{Worst case scenario.}} looks like:}
\bgroup\large
\only<4>{
\begin{table}
{\setlength{\extrarowheight}{5pt}
\begin{tabular}{ccccccccc}
$\color{olive}{\tilde{u}_L}$&$\color{olive}{\tilde{u}_R}$&$\color{olive}{\tilde{c}_L}$&$\color{olive}{\tilde{c}_R}$&$\color{olive}{\tilde{t}_L}$&$\color{olive}{\tilde{t}_R}$&&$\color{red}{\tilde{B}}$&$\color{red}{\tilde{W}^0}$\\
$\color{cyan}{\tilde{d}_L}$&$\color{cyan}{\tilde{d}_R}$&$\color{cyan}{\tilde{s}_L}$&$\color{cyan}{\tilde{s}_R}$&$\color{cyan}{\tilde{b}_L}$&$\color{cyan}{\tilde{b}_R}$&&$\tilde{g}$&$\color{orange}{\tilde{W}^\pm}$\vspace{.75em}\\
$\color{purple}{\tilde{e}_L}$&$\color{purple}{\tilde{e}_R}$&$\color{purple}{\tilde{\mu}_L}$&$\color{purple}{\tilde{\mu}_R}$&$\color{purple}{\tilde{\tau}_L}$&$\color{purple}{\tilde{\tau}_R}$&&$\color{orange}{\tilde{H}_d^-}$&$\color{red}{\tilde{H}^0_d}$\\
$\color{brown}{\tilde{\nu}_e}$&&$\color{brown}{\tilde{\nu}_\mu}$&&$\color{brown}{\tilde{\nu}_\tau}$&&&$\color{red}{\tilde{H}_u^0}$&$\color{orange}{\tilde{H}^+_u}$
\end{tabular}
}\end{table}
}
\only<5>{
\begin{table}
{\setlength{\extrarowheight}{5pt}
\begin{tabular}{ccccccccc}
$\color{olive}{\tilde{u}_1}$&$\color{olive}{\tilde{u}_2}$&$\color{olive}{\tilde{u}_3}$&$\color{olive}{\tilde{u}_4}$&$\color{olive}{\tilde{u}_5}$&$\color{olive}{\tilde{u}_6}$&&$\color{orange}{\tilde{\chi}^\pm_1}$&$\color{orange}{\tilde{\chi}^\pm_2}$\\
$\color{cyan}{\tilde{d}_1}$&$\color{cyan}{\tilde{d}_2}$&$\color{cyan}{\tilde{d}_3}$&$\color{cyan}{\tilde{d}_4}$&$\color{cyan}{\tilde{d}_5}$&$\color{cyan}{\tilde{d}_6}$&&$\tilde{g}$&\vspace{.75em}\\
$\color{purple}{\tilde{\ell}_1}$&$\color{purple}{\tilde{\ell}_2}$&$\color{purple}{\tilde{\ell}_3}$&$\color{purple}{\tilde{\ell}_4}$&$\color{purple}{\tilde{\ell}_5}$&$\color{purple}{\tilde{\ell}_6}$&&$\color{red}{\tilde{\chi}^0_1}$&$\color{red}{\tilde{\chi}^0_2}$\\
$\color{brown}{\tilde{\nu}_1}$&&$\color{brown}{\tilde{\nu}_2}$&&$\color{brown}{\tilde{\nu}_3}$&&&$\color{red}{\tilde{\chi}^0_3}$&$\color{red}{\tilde{\chi}^0_4}$
\end{tabular}
}\end{table}
}
\egroup
\only<6>{
Summary of mixed states:
\begin{itemize}
\item The Higgs bosons ($H_d^-$, $H_d^0$, $H_u^0$, $H_u^+$) form: a charged scalar pair $H^\pm$, two neutral scalars $h^0$, $H^0$, and a neutral pseudoscalar $A^0$.
\item The charged bosinos ($\tilde{W}^\pm$, $\tilde{H}_u^+$, $\tilde{H}_d^-$) form the charginos $\tilde{\chi}^\pm_i$.
\item The neutral bosinos ($\tilde{B}$, $\tilde{W}^0$, $\tilde{H}_u^0$, $\tilde{H}_d^0$) form the neutralinos $\tilde{\chi}^0_i$.
\item The squarks ($\tilde{q}_{i, L}$, $\tilde{q}_{i, R}$) form mass eigenstates labeled $\tilde{q}_i$.
\item The charged sleptons ($\tilde{e}_{i, L}$, $\tilde{e}_{i, R}$) form eigenstates $\tilde{\ell}_i$.
\item The sneutrinos $\tilde{\nu}_i$ form eigenstates $\tilde{\nu}_i$.
\end{itemize}
Only for certain ranges of the parameters the particle spectrum will resemble a 'double-SM'.
}
\end{frame}
\subsection{Parameter Count}
\begin{frame}{MSSM — Parameter count}
\only<1, 2, 3, 4>{$+$
\begin{itemize}
\item 3 couplings $g_i$, one vacuum angle $\theta_\mathrm{QCD}$ (4 parameters)
\item 3 (complex) gaugino masses $M_i$ (6 parameters)
\item 2 Higgs mass parameters $v$, $\beta$ (2 parameters)
\item 2 (complex) Higgs/ino mass parameters $\mu$, $B$ (4 parameters)
\item 5 hermitian scalar mass matrices $M^2_{\tilde{F}}$ (5x9 parameters) 
\item 3 mass matrices $\lambda^f$ (3x18 parameters)
\item 3 trilinear couplings $T_F$ (3x18 parameters)
\end{itemize}
}\only<2, 3, 4>{$-$
\begin{itemize}
\item Flavour symmetry $\un(3)^5/\un(1)^2$ (5x9-2 parameters)
\end{itemize}
}\only<3, 4>{$+$
\begin{itemize}
\item R- and Peccei-Quinn symmetry $\un(1)_R\times\un(1)_\mathrm{PQ}$ (2 parameters)
\end{itemize}}
\vfill
\only<4>{The full Minimal Supersymmetric Standard Model has \alert{124 free parameters}\footnote{Consisting of 3 couplings, 37 real masses, 39 mixing angles and 45 CP-violating phases.} (MSSM-124).}
%105 new parameters 5r 3cp gaug higgsino, 21 sfermion masses, 36 mix angl, 40 cp
\end{frame}