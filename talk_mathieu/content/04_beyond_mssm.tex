\section{Going Beyond the MSSM}

\begin{frame}{Are all Problems solved now?}
\addtocounter{framenumber}{-1}
	The MSSM as introduced in the first part of our talk still fails to answer some of the key questions:\\[1em]

\begin{itemize}
	\item How do we explain the value of the $\mu$ parameter in the MSSM?\\[1em]
	\item What about the \enquote{total} particle content?\\[1em]
	\item Is there a \enquote{natural} way to implement SUSY-breaking?\\[1em]
	\item What is $\mathcal{G}_{\mathrm{MSSM}}$?\\[1em]
	\item and (many) more \dots\\[1em]
 \end{itemize}
To conclude our talk we want to have a look at some ideas concerning the first mentioned problem, the value of the $\mu$ parameter. 
\end{frame}
\begin{frame}{Different Approaches to explain the Value of the $\mu$ Parameter}
\begin{itemize}
\item \textbf{General Problem:} $\mu$ is a \alert{SUSY-preserving parameter}, but from phenomenology we know that it must be of the order of the SUSY-breaking scale.\\[1em]

	\item \enquote{Natural} solution: Symmetry enforcing $\mu=0$, and a small SUSY-breaking parameter that generates a value of $\mu$ which is not parametrically larger than the SUSY-breaking scale.\\[1em]
\end{itemize}
In proposed extensions of the MSSM, some other approaches have been presented, for example:\\[1em]
\begin{enumerate}
	\pause
	\item Replace $\mu$ by the VEV of a new $ \mathrm{SU}(3) \times \mathrm{SU}(2) \times \mathrm{U}(1)$ scalar singlet $\implies$ NMSSM\\[1em]
	\pause
	\item Add a new broken $\mathrm{U}(1)$ gauge symmetry to the NMSSM $\implies$ USSM\\[1em]
	\pause
	\item Allow all possible renormalizable terms in the superpotential $\implies$ new mass terms (GNMSSM)\\[1em]
	\pause
	\item Possible connection to the strong CP problem? $\implies$  PQ symmetry $\implies$ Axion Physics?\\[1em]
	\pause
	\item Higher-dimensional Higgs multiplets?
\end{enumerate}
\end{frame}

