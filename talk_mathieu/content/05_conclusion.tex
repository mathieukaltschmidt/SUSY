\section{Summary and Outlook}
\begin{frame}{What have we learned today?}
\addtocounter{framenumber}{-1}
	\begin{itemize}
		\item The \alert{SM can be promoted to the MSSM} using the concepts and methods introduced in the scope of this seminar.\\[1em]
		\item In addition to the 19 parameters of the SM we get $\mathcal{O}(100)$  new ones, which complicates the experimental access to the MSSM a lot! \\[1em]
		\item Phenomenological models of the MSSM allow for a \alert{drastic reduction of the number of independent parameters} that have to be measured experimentally.\\[1em] 
		\item SUSY may lead to a better understanding of the \alert{connection between Gravity and the SM}. Additionally SUGRA helps us to further reduce the number of dof's in the MSSM.\\[1em] 
		\item In the MSSM the \alert{unification of the SM gauge couplings} at some high-energy scale $M_X$ can be realized due to the effect of loop-corrections arising from the additional superpartners.\\[1em] 
		\item There are still plenty of problems where the MSSM fails to explain the experimental results. \\[0.5em] $\implies$ The MSSM is not the \enquote{ultimate} model to describe our Universe.
	\end{itemize}
\end{frame}